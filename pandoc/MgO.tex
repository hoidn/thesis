
\emph{This chapter was prepared as an article for submission to Physical Review B (currently in preparation)}


\author[1]{aaaab}

\affil[1]{afiil}

{\centering
Oliver Hoidn\textsuperscript{1}, Ryan Valenza\textsuperscript{1}, Gerald
T. Seidler\textsuperscript{1,(*)}\textsubscript{,} Alexander
Ditter\textsuperscript{1}, William Holden\textsuperscript{1}, Evan
Jahrman\textsuperscript{1}, Samuel Vinko\textsuperscript{2}, Josh
Kas\textsuperscript{1}, Fer Vila\textsuperscript{1}, Alison M.
Saunders\textsuperscript{3}, Luis Avila\textsuperscript{4}, Galen
O'Neil\textsuperscript{4}, Hae Ja Lee\textsuperscript{5}, Bob
Nagler\textsuperscript{5}

\bigbreak

\textsuperscript{1}Department of Physics , University of Washington,
Seattle WA 98122

\textsuperscript{2} Department of Physics, Clarendon Laboratory, University of Oxford, Parks Road, Oxford OX1 3PU, UK

\textsuperscript{3}Department of Physics, University of California,
Berkeley CA 94720

\textsuperscript{4}National Institute of Standards and Technology,
Boulder CO 80305

\textsuperscript{5}Stanford Linear Accelerator Center, Menlo Park CA
94025

}

\bigbreak

\begin{addmargin}[4em]{1em}
We report a study of the x-ray diffraction of nanocrystalline MgO during
heating by high-intensity x-ray pulses from an x-ray free electron laser
(XFEL). Careful consideration of the three lowest angle Bragg peaks
gives strong evidence that the MgO remains crystalline during the XFEL
pulse. As a consequence of the purely constructive or purely destructive
interference between unit cell sites for the different Bragg peaks, we
are able to measure the the difference of nominal ionization of the Mg
and O crystallographic sites as a function of incident flux density. We
find an anomalously low threshold for the onset of substantial O
2\emph{p} ionization, requiring that the intense x-ray illumination
induces a large density of states in the MgO ground-state energy gap.
Unlike usual ionization potential depression effects resulting from
screening, we propose that the early onset of O 2p ionization may be a
consequence of high site disorder of the charge state of the Mg and O
ions, giving a distinct, valence-level mechanism for nonlinear response
to x-ray heating. Unlike core-level x-ray nonlinearities that occur due
to lifetime effects, i.e., bleaching, the present nonlinearity is a
consequence of the changes in the electronic potentials experienced by
the free or weakly bound charge density. In an attempt to elucidate the
likely dominant mechanisms in the heating and relaxation cascades, the
dependence of Bragg peak intensities over the available range of
incident flux density results are compared and contrasted with a variety
of theories spanning usual approached in solid state physics, molecular
chemistry, and low-density plasma physics
\end{addmargin}

(*) seidler@uw.edu

\section{I. Introduction}

The `warm dense matter' (WDM) regime resides in an important and
theoretically interesting middle-ground between the conditions typical
of high-pressure studies in condensed matter physics and those instead
of great interest to traditional plasma physics, where atoms are fully
ionized. WDM is defined by partial ionization, solid-like or higher
densities, substantial Fermi degeneracy (\(T \leq \ E_{F}\)), and values
for the plasma coupling parameter Γ of order unity or larger. {[}1{]}
Beyond intrinsic interest in understanding this exotic physical regime,
the collective properties of WDM---such as its equations of state (EOS),
opacities, and viscosities---are of fundamental importance in
geophysics, planetary and stellar astrophysics, and inertial confinement
fusion (ICF). {[}2,3{]}

WDM, however, presents unique practical challenges. While some obvious
difficulties accrue from the details of how this transient state of
matter is actually created, hard problems arise from the question of
experimental diagnostics. Specifically, the high opacity of dense,
partially-ionized matter renders ineffective the well-developed suite of
optical-wavelength tools that are regularly used to obtain detailed
microscopic information for low density plasmas. The community has
therefore turned to x-ray wavelength diagnostics, each of which has a
different connection to microscopic observables of interest. X-ray
absorption spectroscopy interrogates the unoccupied electronic density
of states, x-ray fluorescence informs us about semi-core and core-level
occupancies of the ions, and elastic and inelastic x-ray scattering
characterize the real-space charge density and momentum-space
distribution functions, respectively. {[}4-6{]}

The role of elastic x-ray scattering, i.e., x-ray diffraction (XRD), in
the study of crystalline WDM has recently been discussed by Valenza and
Seidler. {[}7{]} While all early XRD studies of WDM were performed in
laser-shock studies where long-range crystalline order of the target is
destroyed, those authors instead address the question of XRD from WDM
created by heating of crystalline targets by the extremely fast pulses
created by x-ray free electron lasers (XFELs). {[}8{]} While one key
result of that work is the large contribution to XRD of the nominally
`free' electrons for low-Z systems, there are two overarching
perspectives in that work that help motivate the present study. First,
XRD from crystalline WDM provides an important testing ground for
finite-T electronic structure theory, giving direct measurement of the
fundamental quantity predicted in DFT approaches, i.e., the spatial
distribution of charge density. For example, for low-Z systems, finite-T
studies of XRD might be used to give an especially salient inquiry into
exchange functional effects under intermediate degeneracy of the unbound
electrons. Second, and of greater relevance here, XRD from a WDM state
where the ion cores are at least effectively stationary is a rich
experiment that, through the careful choice of target material crystal
structure, can be designed to have far higher information content about
microscopic parameters, such as species-specific ionization states, than
has been the case in studies of disordered WDM. In particular, we show
here that high ionization state sensitivity can be obtained by
judiciously selecting a crystalline compound whose diffraction peaks
express a combination of destructive and constructive interference
between different atomic sites in the unit cell.

We report here a study of the XRD from nanophase XFEL-heated targets of
the simple compound MgO. In the context of the above discussion, this
choice is highly desirable. First, MgO, which has the NaCl rock-salt
structure, is an example of the simplest crystal structure where there
are low-angle Bragg peaks that have either purely constructive or purely
destructive interference between the dissimilar species in the unit
cell. This property is nontrivial, and it gives the experiment
sensitivity to not just the average ionization across the unit cell but
also to the ionization of each species. Second, the combination of
slightly heavier atoms than in the Valenza {[}7{]} work together with
favorably distributed conduction-band charge densities allows
interpretation in the context of a simple atomic form factor (AFF) model
that ignores the contributions to XRD from the unbound electrons. Third,
the long semicore lifetime of Mg 2p holes provides an interesting
problem for the present work, where there is then a meaningful
competition between O valence-level and Mg 2p-level depopulation, but
also provides a preludeto the future where variable-delay x-ray
pump/probe studies will be able to measure systems' responses to
long-lived electronic excitations with fine time resolution compared to
durations of the nonequilibrium states involved. Finally, while the
nanophase nature of the target poses some added complications in
calculating the average energy actually deposited in each unit cell, it
ensures good sample isotropy so that the XRD rings are uniform and
ensemble-averaged for each single shot. This averts a problem that can
arise in coarse-grained powder diffraction samples, wherein a relatively
small number of crystallites in the probed sample volume results in
insufficient sampling of the distribution of random crystallite
orientations.

During the single-shot studies where heating and diffraction occur
simultaneously, we find no signatures of long-range lattice disorder,
such as Deybe-Waller or `Bragg gating' effects, across a range of energy
deposition densities reaching 150 eV per unit cell. However, we measure
a monotonic rise in intensity of the (111) peak of MgO with increasing
XFEL flux density. This effect is the fully expected signature of of a
loss of destructive interference within the MgO unit cell as the valence
electrons of O progressively delocalize. This is an observation of
short-range, purely electronic charge reorganization in a solid-density,
partially-ionized plasma that constitutes an initial demonstration of
the general type of `warm dense crystallographic' effect predicted by
Valenza, et al. {[}7{]} The observed onset for O 2p ionization is,
however, anomalously low when compared to constraints that would be
imposed by the large ground-state band gap of nearly 8 eV. We propose
that the site-disorder of ionization that is a hallmark of x-ray heating
is likely resulting in a great enhancement of states in the ground state
band gap, thus giving a new mechanism, independent of, e.g., traditional
inonization potential depression (IPD), for enhanced ionization effects
in WDM. Such `Lifshitz tails' are well known in the theory of
semiconductor physics but have not previously been discussed in the
context of WDM. Although the lack of local thermal equilibrium (LTE) in
the present study limits the utility of comparisons to theory, the
experiment validates wide-angle XRD as an effective probe of local
real-space electronic reorganization in crystalline warm dense matter.
It thus presents the appealing prospect of future XRD studies on
XFEL-heated WDM and solid-density plasmas designed to empirically
constrain DFT-based predictions of finite-temperature electronic
structure, especially for experiments performed using two-color x-ray
pump, x-ray probe methods where many limitations of the present study
will be ameliorated.

We continue as follows. First, in section II we describe experimental
methods. In section III we discuss approaches for modeling variations in
the Bragg diffraction signal as a consequence of XFEL heating. This
includes ground-state, molecular, density functional theory, and
species-by-species radiative combination calculations. In section IV we
present and discuss experimental and modeling results, the most
consequential of which is that the presence of valence disorder
substantially complicates interpretation of WDM structure by
invalidating ground state-based treatments of the electronic structure
and providing a new route for effective enhancement of ionization
effects that is specific to crystalline dense plasmas.. Finally, in
section IV we conclude and discuss future directions.

\section{II. Experimental Methods}

\subsection{II. A. Experimental Details}

The experiment was performed at the matter of extreme conditions (MEC)
endstation of the Linac Coherent Light Source (LCLS), where XFEL pulses
were used to excite MgO samples consisting of a 100 nm-thick layer of
PMMA with embedded MgO nanoparticles (Sigma Aldrich, typical size 50 nm)
on an 8 µm-thick polyimide substrate. We used self-amplified stimulated
emission (SASE) pulses of 45 fs mean duration, average pulse energies of
2 mJ, and a nominal x-ray energy of 9 keV. Variations in the mean photon
energy of each pulse are monitored by a downstream dispersive
spectrometer. We controlled flux density incident on the sample via a
stack of Be lenses, with which we varied the focal spot diameter of XFEL
pulses at the sample position from 2 to 60 µm; these diameters were
determined using an ablative imprint method to measure the spatial
profile of the focused XFEL beam. {[}9{]} Using the full available range
of focal spot sizes and the unattenuated beam, we obtained incident flux
densities ranging from 30 to 2000 J/cm\textsuperscript{2}.

During data collection the sample's position was rastered at a rate of
100 µm between the XFEL pulses, whose repetition rate was 120 Hz. At
every XFEL pulse the Bragg scattering signal was collected and read out
from a quad CSPAD solid state detector having 800 x 800 resolution and a
pixel pitch of 100 µm. {[}10,11{]} The 8 x 8 cm\textsuperscript{2}
active area of the detector subtended the range of scattering angles
from 10 to 58 degrees, encompassing the (111), (200), and (220) Bragg
reflections of MgO (located at 33.5, 38.8, and 55.9 degrees,
respectively, for 9 keV incident photons).

\subsection{II.B. Data Reduction and Analysis}

Several steps of processing and event selection were performed prior to
generating powder diffraction patterns. For each event the quad CSPAD
readout was corrected by subtracting pixel pedestals (measured using
previously-collected dark exposures) as well as common-mode noise in
each of the detector's 16 individual tiles. {[}10,11{]} Due to the small
number of photon hits in a single shot residual ADC noise often
dominated the Bragg scattering signal. To address this we made use of a
standard component of the analysis pipeline for CSPAD data in low-photon
count rate experiments at the LCLS, such as macromolecular imaging and
crystallography, which identifies clusters of spatially-concentrated
signal resulting from x-ray photon hits, rejecting the output of
noise-dominated pixels. {[}12{]}

For each LCLS pulse a powder diffraction pattern is generated from the
quad CSPAD frame by the summation of elliptically-shaped strips of
pixels at equal scattering angle. The mapping of pixel coordinate to
scattering angle is calculated from the CSPAD's location and orientation
relative to the sample and incident XFEL beam; this geometry is in turn
obtained from the conic section parameters of powder diffraction peaks
in data from a known reference material measured in the same
source-detector geometry. After generation of a powder pattern, two
corrections are made. First, the each peak is shifted to correct for
angular offsets caused by imperfect flatness of the sample substrate and
also event-to-event jitter in the mean photon energy of the XFEL.
Second, a linear fit is made to the background of each peak, and this
background is subtracted from the peak signal. The signal-to-background
ratio for this peak is then computed by comparison of the background
level derived from the linear fit to a simple integration of the
background-subtracted peak. Events in which any of the peak
signal-to-background ratios fall below a threshold (chosen to be 0.2)
are rejected in subsequent summation over data from multiple events.

Bragg peak scattering intensity is the most significant derived quantity
from each powder pattern in this study, but its estimation requires
correcting the effect of variations in the total scattering signal
caused by sample non-uniformity and temporal variation of the XFEL pulse
energy. The latter contribution can be corrected using direct
measurement of incident pulse intensities available from an upstream
nitrogen detector; for comparison of Bragg peak intensities at different
flux density values, however, we normalize each pattern to the integral
intensity of its (200) peak in order to control for variations in sample
thickness, or fluctuations in nanoparticle volume in the beam, such as
from nonuniform aggregation..

\subsection{III. Modeling Methods}

Finally, the relationship between the incident flux density and the
resultant energy density in the MgO nanoparticles requires some care.
The small (100 nm) sample thickness requires that a significant portion
of higher-energy electrons created in the relaxation cascade following,
e.g., primary photoionization of the Mg 1s orbital, will necessarily
escape into the surrounding low-Z substrate and surrounding binder,
causing a reduction in the density of \emph{deposited,} versus absorbed,
energy. This effect has recently been discussed in detail, and proposed
to be especially important for the design of XFEL x-ray heating targets.
In the present case, using the methods described in {[}13{]} we use
PENELOPE to perform Monte Carlo simulations for 9 keV incident photons
striking a target consisting of a 50-nm thick MgO layer clad with
graphite, where graphite is taken as representative of carbonaceous
binder and substrate materials. This simulation implements
particle-tracking simulation of electron showers in which elastic
scattering differential cross sections are calculated from partial-wave
solutions to the Dirac equation, while inelastic interactions (involving
both impact ionization and collective excitations) are represented using
a modified version of Liljequist's `delta-oscillator' generalized
oscillator strength (GOS) model. {[}14{]}

The average energy deposited per unit cell is then calculated on the
basis of the incident pulse energy, the focal spot size, standard (cold)
cross sections for x-ray absorption, which greatly dominates Compton and
elastic scattering, and the above correction. This calculation is an
upper bound, in that it assumes that long-wavelength electronic
excitations, e.g., plasmons, are a minor contributor to the energy
distribution at any moment in the relaxation cascade and will have
decayed to simple electron-hole excitations during the time duration of
the pulse.

To model the dependence of the XRD signal on the electronic
configurations of the Mg and O ions we use the Hartree-Fock code of
Cowan {[}15{]} to calculate the atomic form factors of Mg and O
decomposed by subshell. The crystal's structure factor is subsequently
calculated as a sum over basis atoms and subshells; i.e.,

\(S\left( \overrightarrow{Q} \right) = \sum_{j}^{}{\sum_{n,l}N_{n,l}f_{j,n,l}\left( Q \right)e^{- i\ \overrightarrow{Q} \bullet \overrightarrow{r_{j}}}\ },\)
(1)

where \(f_{j,n,l}\left( Q \right)\) is the atomic form factor of the
subshell \(n,l\) of the jth species, where \(n\) and \(l\) are principal
and orbital angular momentum quantum numbers, and \(N_{n,l}\) is the
subshell's population. The intensity of a given Bragg reflection
(neglecting Debye-Waller quenching and the geometric dependence of
scattering by a powder of crystallites) is then obtained by evaluating
\(S\left( \text{h\ }\mathbf{a}_{\mathbf{1}} + \ k\ \mathbf{a}_{\mathbf{2}}\mathbf{+ \ }\text{l\ }\mathbf{a}_{\mathbf{3}} \right)\),
where \(\mathbf{a}_{\mathbf{1}}\mathbf{,\ }\mathbf{a}_{\mathbf{2}}\) and
\(\mathbf{a}_{\mathbf{3}}\) are the basis vectors of the reciprocal
lattice and \(h\), \(k\) and \(l\) are Miller indices. Following
standard practice in plasma physics modeling, we treat ionization of
atomic electrons as a uniform (real-space) smearing of free electrons;
thus, ionization of an atomic orbital simply corresponds to reduction of
its weight \(N_{n,l}\).

In the high-energy density regime, defined as having mean temperatures
above 2 eV, we simulated the temporal evolution of the MgO charge state
distribution (CSD) over the course of an XFEL pulse using a variant of
the collisional radiative code SCFLY that Vinko et al. have modified to
self-consistently support elemental mixtures. {[}16{]} The code
implements a local density-based treatment of continuum lowering that
Vinko et al. have demonstrated accurately reproduces experimental
ionization potential shifts at high charge states in several
solid-density plasma mixtures. Unlike other collisional radiative codes,
SCFLY models the plasma's free electron energy density self-consistently
with respect to XFEL heating and interaction with ions (e.g. impact
ionization and Auger decay). Its principal caveat in the current
setting, where the plasma is heated via photoexcitation by photons with
energies far above the absorption edges of Mg and O, is its assumption
of an instantly-equilibrating thermal distribution of free electrons.
{[}17{]}

Our inputs to SCFLY were the sample composition and XFEL photon energy,
flux density, and temporal profile (which we took as a Gaussian with a
45-fs FWHM duration); the main outputs were temporal profiles of the
charge state populations of Mg and O during the XFEL pulse. With the
help of a simplifying approximation (discussed in section III below)
that relates Mg and O charge states to 2p ionization, we used the AFF
ionization model to obtain predicted Bragg peak intensity ratios for
each simulated incident flux density.

In the low-energy density regime we instead adopt three alternate models
that capture aspects of the condensed phase physics under contrasting
assumptions and limiting conditions. First, we use the Vienna ab-initio
Simulation Package (VASP) {[}18{]}, a density functional theory (DFT)
code, to compute the X-ray diffraction signal at finite temperature
using the charge distributions of Kohn-Sham eigenstates populated by
Fermi-Dirac Statistics. The relation between temperature and deposited
energy density is derived from the zero-temperature density of states of
MgO, which we compute using the code FEFF. {[}19{]} Under this model it
is assumed that the potential landscape and density of states are not
significantly altered by finite-temperature redistribution of charge.
Additionally, the presence of local thermal equilibrium is implicit.

Second, we take a simplified picture of XFEL heating where all energy
deposited in the MgO sample contributes to excitation of O 2p states.
Departing from the ground state density of states (DOS), we assume that
the energy required to delocalize an O \emph{2p} electron is equal to
the band gap, 7.8 eV. Under these assumptions a given XFEL dose
therefore generates a known amount of O 2p ionization; from this, the
AFF model is used to calculate the resulting XRD response. We term this
approach the ground state model.

Third and finally, we consider the limit in which perturbation of the
electronic structure creates a large density of states in the band gap
and the O \emph{2p} ionization potential is determined by local
interactions alone. In this molecular model we approximate the
ionization potential of O\textsuperscript{2-} using a delta SCF
(self-consistent field) calculation. {[}20{]} This approximation yields
an ionization potential of 1.9 eV, from which the XRD response can be
calculated in the same fashion as in the ground state model

\section{III. Results and Discussion}

\begin{center}
\includegraphics{MgO_2.2.docx1502866987/media/image1.png}
\end{center}
\textbf{Fig. 1}. (a): Experimental intensity of the (220) Bragg peak
from a sample of MgO simultaneously heated and probed by 45-fs duration
XFEL pulses, as a function of XFEL energy deposited per MgO unit cell,
with no normalization across individual data points. (b): Equivalent
data for the (200) peak. (c): Experimental intensities of the (111)
peak, compared to several models. Each experimental data point is
normalized by (1) its intensity at minimum flux density and (2) the
(200) peak intensity. For each of the four displayed models, the shaded
region corresponds to the locus of possible curves once the loss of
in-sample energy deposition due to nonlocal heat transport by hot
electrons is accounted for. See the text for discussion.
\bigbreak

To begin, in Fig. 1 (a), we show the experimentally measured intensity
of the MgO (200) peak as a function of energy deposited per MgO unit
cell. The \textasciitilde{}15\% scatter in the observed scattering
intensity upon increasing excitation is explained as being due to
variations in MgO nanoparticle content across different regions of the
sample. Consequently, our first result is clear: the MgO nanoparticles
remained substantially, and possible completely, crystalline for the
duration of the XFEL pulse. There is no evidence for `Bragg-gating' or
other self-limiting diffraction signals that are known to be important
in the context of macromolecular crystallography at XFELs. {[}21{]}

In Fig. 1 (b) we plot the normalized experimental intensity of the (111)
and (220) Bragg peaks of MgO as a function of incident flux density
(together with curves for several models, which we discuss below).
Specifically, for each Bragg peak, the entire curve is normalized to the
intensity of the ``cold'' (lowest-flux density) dataset and each
individual data point is normalized to the (200) peak intensity for the
corresponding flux density. The normalization to the intensity of the
(200) peak helps to remove fluctuations in diffracted intensity due to
sample thickness nonuniformity. We hereafter refer to intensities
normalized in this fashion, for a given peak (hkl), as
\(I_{\text{hkl}}/I_{200}\). Displayed error bars are estimated
systematic errors due to background subtraction and peak integration;
counting statistics-derived errors are negligible. The most salient
feature is a 20\% rise in the relative intensity of the (111) peak
between the lowest and highest flux densities. In contrast, the relative
intensity of the (220) peak fluctuates but does not display a monotonic
progression. The behavior of both curves---and in particular the rise in
relative intensity of the (111) peak---is strongly at odds with the any
Bragg peak quenching that would result from a Debye-Waller thermal-like,
uncorrelated, increase in the mean squared displacement of atoms from
their lattice sites. The absence of any such signature further supports
the crystallinity of the heated target and the isolation of the
deposited energy in the electronic, rather than lattice, degrees of
freedom.

A first step toward understanding the increase in relative intensity of
the (111) Bragg peak comes from consideration of the ground-state x-ray
crystallography of MgO. MgO's rock salt-type crystal structure consists
of two interpenetrating FCC lattices of Mg and O, with one of the
lattices shifted by half of the FCC lattice constant in the direction of
one of the lattice basis vectors. This has consequences for the
dependence of the (200), (111), and (220) Bragg peak intensities on the
characteristics of the ions on the two sites in the primitive basis, as
is frequently discussed in introductory texts. {[}22{]} In particular,
the (200) and (220) peaks result from perfect constructive interference
between the two unit cell sites while the (111) peak, on the other hand,
instead has perfect destructive interference between the two unit cell
site. The nominal ground-state ionic species of MgO,
Mg\textsuperscript{2+} and O\textsuperscript{2-}, have identical
electron configurations and have only very slightly different ionic form
factors for x-ray scattering as a secondary consequence of the different
nuclear potentials on the spatial extent of the electronic
wavefunctions. This offers an explanation for the small ground-state
intensity of the (111) Bragg peak, as well as for its monotonic rise
with increasing incident x-ray flux: as temperature increases electrons
of the weakly-bound O 2p orbitals are ionized at a higher rate than
those of Mg semi-core 2p orbitals, increasing the dissimilarity of the
form factors of the O and Mg ions.

This relationship is illustrated by Fig. 2, which shows
\(I_{111}/I_{200}\) as a function of O 2\emph{p} and Mg 2\emph{p}
population under the AFF ionization model presented in section II. Three
curves denoting different scenarios for the electronic configuration of
Mg and O following x-ray heating are indicated in Fig. 2b, and the
corresponding Bragg peak intensity progressions are plotted in Fig. 2b.
First, trajectory (i) corresponds to progressive ionization of the O 2p
electrons with a fixed (ground state) population of Mg 2\emph{p}
electrons; it encompasses possible states of the x-ray excited system
following relaxation of all deeply-bound Mg 2\emph{p} holes. Along
trajectory (i), removal of O 2p charge density causes the (111) Bragg
peak intensity to increase monotonically due to reduction in the O
electrons' cancellation of the Mg charge density's scattering amplitude.
Conversely, on trajectory (ii) only Mg 2\emph{p} electrons are ionized;
noting that the Mg K shell dominates the photoelectric cross section of
MgO at hard x-ray energies, this corresponds to the locus of possible
initial states following instantaneous x-ray ionization. In this case,
ionizing the Mg 2p orbitals (starting from the ground state) reduces the
magnitude of the total scattering amplitude until total destructive
interference is reached before the scattering factors differ again as Mg
2\emph{p} ionization continues. The resulting intensity progression for
case (ii) in Fig. 2b therefore has a local minimum (located at an Mg
2\emph{p} population of approximately 3.5 electrons). Finally,
trajectory (iii) represents an intermediate scenario wherein the levels
of O and Mg 2\emph{p} ionization are (artificially) equal. Here, the
dominant effect is a slow decrease in intensity due to uniform reduction
of the scattering amplitude contributions of all 2p electrons in the
unit cell.

\begin{center}
\includegraphics{MgO_2.2.docx1502866987/media/image2.png}
\end{center}
\textbf{Fig. 2}. Dependence of
\hyperdef{}{OLEux5fLINK16}{}{\hyperdef{}{OLEux5fLINK17}{}{\hyperdef{}{OLEux5fLINK18}{}{}}}\(I_{111}/I_{200}\)
on population of the \emph{2p} orbitals of
\hyperdef{}{OLEux5fLINK14}{}{\hyperdef{}{OLEux5fLINK15}{}{}}\(\text{Mg}^{2 +}\)
and \(O^{2 -}\), according to an atomic form factor (AFF) based model of
ionization. This ratio reaches a maximum factor of 13.8 times the
unperturbed value at full ionization of the O 2p electrons.
\bigbreak

Referring back to the results of Fig, 1
for\hyperdef{}{OLEux5fLINK1}{}{\hyperdef{}{OLEux5fLINK2}{}{\hyperdef{}{OLEux5fLINK5}{}{}}}
\(I_{111}/I_{200}\), we see a more gradual enhancement than is the case
for a thermalized system with only O 2\emph{p} ionization and there is
no evidence of initial decreases for states dominated by Mg 2\emph{p}
ionization. The resulting qualitative conclusion is then a trajectory
intermediate between (i) and (ii), where any theoretical treatment will
need to include both the lifetime of the Mg 2\emph{p} core holes during
the x-ray excitation and also the presence and significant, if
incomplete, thermalization of O 2\emph{p} valence-level electrons.

\FloatBarrier

Hence, predicting the consequences of single-pulse XFEL heating on XRD
requires separate calculation of an average of the probed Mg and O
ionization states ---both temporal (over the duration of an XFEL pulse)
and spatial (over all probed unit cells). We begin with the low-energy
limit. In the regime where energy deposition per unit cell is less than
1 eV, condensed matter physics and the details of valence-level
electronic structure are predominant. For example, prior work on the XRD
from KH2PO4 after strong optical excitation from the valence band to the
conduction band could be well-interpreted in the general arena defined
by the ground-state electronic structure of the crystalline phase.
{[}23{]}

Here, the absence of local thermal equilibrium (LTE) is a challenge for
modeling fs-scale electronic reorganization in the 0.1-1eV temperature
regime because---unlike in the plasma limit, where the atomic kinetics
are unambiguous (modulo treatment of the ionization potential
depression)---there is a lack of established frameworks for calculating
the time-evolved electronic structure. We thus forgo \emph{ab initio}
simulation and take a simple assumption of proportionality between the
density of deposited energy and the level of O 2\emph{p} ionization.
Under this assumption, we then consider two idealized bounds on the
excitation kinetics.

First, we consider the case where the ground state electronic structure
is taken as a static venue that is unperturbed by even relatively high
levels of ionization, i.e., where the energy needed to excite an O 2p
electron is equal to the band gap of MgO, \(E_{g}\) = 7.8 eV no matter
the level of O 2\emph{p} ionization. The results of this naïve ground
state model are shown as a shaded region (orange) with label AFF 7.8 eV
in Fig. 1. Under this condition, and defining \(\rho_{E}\) as the
density of deposited XFEL energy, it follows that \({\rho_{E}/E}_{g}\)
is an upper bound on the concentration of O 2p excitations (i.e.,
corresponding to all XFEL energy coupled into O 2p excitation). The
yellow shaded region of Fig. 1 shows the ground state model's predicted
intensity progression of the (111) Bragg reflection as a consequence of
a density of deposited energy equal to \({\rho_{E}/E}_{g}.\) The
measured onset of the (111) peak intensity's rise is much delayed
compared to the model's prediction, from which we can infer that the
experimental level of O 2p delocalization is significantly higher than
that allowed by the ground state density of states of MgO -- in other
words, there must be a plethora of states in the band gap that occur as
a consequence of the x-ray excitation, even while periodicity of the
ion-core locations is preserved. This inconsistency is corroborated by
comparison of the experimental progression with a more advanced
calculation of the same ilk, a finite-temperature DFT-based calculation,
which also fails to reproduce the (111) peak's early rise (teal region
in Fig. 1, labeled VASP).

Second, as an alternative bounding case, an isolated, molecular
Mg\textsuperscript{2+}O\textsuperscript{2-} system with a much lower,
calculated ionization threshold of 1.9 eVcan be considered. The general,
quadratic shape is still shown (color, label in Fig. 1), but this lowest
possible ionization threshold results in an overestimate of the onset
for O 2\emph{p} ionization. Hence, the dilute-plasma limit, modified
only by ion-pairing for local neutrality, has omitted too much of the
condensed phase physics.

Given the above discussion, the question then arises as to possible
explanations for the observed enhancement of O 2\emph{p} ionization at
lower incident energy densities. Prosaically, some part of this effect
may be due to our choice of nanophase material. Unsurprisingly, the
finite size of MgO nanostructures manifest surface states with energies
inside the ground state band gap. {[}24{]}

However, the magnitude of the present effect requires a more
intrinsically bulk-like behavior. Though the effect superficially
resembles ionization potential depression (IPD), conventional models of
IPD inevitably fail in the low-energy density regime because they treat
variations in the continuum level as a consequence of screening, with
dependence only on the average ionization and ion density of a plasma.
{[}16,25{]} As such, they are incapable of capturing the condensed phase
physics and molecular chemistry that dominate behavior at low energy
densities Here, the key physics may instead be the fact that long-range
electronic order has been destroyed by the site-randomness of ionization
even while perfect long-range order persists for the ion-core locations.
It is known that site disorder in an otherwise perfectly crystalline
solid (for instance lattice vacancies, impurities, or, as in our case,
randomly-distributed electron vacancies) can introduce localized states
with energies inside the band gap, spectra phenomena referred to as
Lifshitz tails. {[}26{]} One particularly celebrated example of this is
Anderson delocalization. {[}27{]} The possibility that this classic idea
in condensed phase physics may find new application in dense plasma
physics is an interesting result that can be further interrogated with,
e.g., large-cluster quantum chemistry calculations or with other
real-space DFT methods where site disorder of ionization state can be
directly manipulated.

We now turn our attention to high energy deposition densities in Fig. 1.
In this regime, treatment of the interaction between atomic and free
electrons, which gives rise to plasmas physics effects such as continuum
lowering, becomes necessary. The above atomic and solid state treatments
become inappropriate---even for the purpose of establishing rough bounds
on the concentration of excitations---and we instead turn to
time-resolved collisional radiative simulation.

\begin{center}
\includegraphics{MgO_2.2.docx1502866987/media/image3.png}
\end{center}
\textbf{Fig. 3}. Electronic temperature evolution during an XFEL pulse
simulated by the radiative collisional code SCFLY. The incident XFEL
photon energy and flux density are 9 keV and 2 x 10\textsuperscript{4}
J/cm\textsuperscript{2}, respectively. The dashed line represents the
temporal profile of the incident XFEL pulse.

\begin{center}
\includegraphics{MgO_2.2.docx1502866987/media/image4.png}
\end{center}

\textbf{Fig. 4}. Evolution of the mean charge states Mg and O of during
an XFEL pulse simulated by the radiative collisional code SCFLY. The
incident XFEL photon energy and flux density are 9 keV and 2 x
10\textsuperscript{4} J/cm\textsuperscript{2}, respectively. The dashed
line represents the temporal profile of the incident XFEL pulse.

The principal outputs of such a simulation are temporal evolutions in
electron temperature and atomic species charge states. Figs. 3 and 4
display these data for simulated XFEL heating of an MgO target,with an
incident XFEL intensity equal to the highest experimental value using
the code SCFLY. Notably, the charge state distribution is strongly
athermal: the difference between the initial and final Mg 2p population
levels is 0.8 electrons, which exceeds the equilibrium ionization level
corresponding to the final temperature of 19 eV by a large factor. One
reason for the lack of LTE is readily apparent. The lifetime of Mg 2p
holes is large compared to the 45 fs XFEL pulse duration {[}28,29{]} and
the simulated free-electronic temperatures are far below the Mg 2p
binding energy. Consequently, the production of Mg 2p holes will be
dominated by rates of 2p and 1s photoionization and electron impact
ionization during the XFEL pulse.

\FloatBarrier

The low value of the free electronic temperature relative to the
\emph{K} shell binding energies of O and Mg allows a simplification in
application of the AFF ionization model to the output of SCFLY. Because
only the 2\emph{p} orbitals of Mg and O are substantially ionized, the
charge state of each species uniquely determines its electronic
configuration under Eq. 1. Progressions of Mg and O charge states (or,
equivalently, Mg and O 2\emph{p} populations) therefore contain
sufficient information to compute Bragg peak intensities using the AFF
model. Fig. 1c shows the output of the resulting SCFLY-based XRD
calculation evaluated over the full range of flux densities simulated
with SCFLY At high flux density agreement between the experimental Bragg
peak intensity data and SCFLY-based model is poor but shares a
qualitative features with the experimental data, namely a rapid rise in
intensity of the (111) peak beyond a 5 eV per unit cell energy
deposition.

A few explanations can be proposed for the plateau in
\(I_{111}/I_{200}\) at higher flux densities, although additional work
will clearly be needed. First, the population of excitons may saturate
at high flux densities due to dependence on the exciton recombination
rate on deposited energy density. A robust modeling of the energy
relaxation cascade that includes both long-wavelength excitations
(plasmons) and also point-like excitations (ionization) would be needed
to better understand such a proposition. Alternatively, a reduction in
the rate of ionization may arise from an increase in the 2\emph{p}
ionization potential as more electrons enter excited states (though this
effect competes with IPD).

\section{IV. Conclusion and Future Directions}

We have explored the use of single hard x-ray XFEL pulses to
simultaneously create and probe crystalline WDM via wide-angle x-ray
diffraction. We present experimental results on the consequences of XFEL
heating on the electronic structure of MgO as a function of deposited
energy density, using a Hartree-Fock orbital-based model of ionization
to infer electronic subshell populations from experimental Bragg peak
intensities. We find that the experimental XRD signal is a sensitive
measure of charge reconfiguration, allowing inference of valence
ionization levels with a precision of under 0.1 electrons per unit cell.
This sensitivity is in large part contingent on the structure of the
system chosen: the odd-numbered Bragg reflections of MgO exhibit
near-destructive interference in the ground state with a rapid,
easily-measurable increase in intensity upon delocalization of electrons
in the highest occupied molecular orbital (HOMO).

The experimental data shows a rapid delocalization of O 2p electrons at
deposited energy densities per unit cell far below the 7.8 eV band gap
of MgO, which constitutes evidence for the creation of excitations
within the ground state band gap. However, interpretation of the data is
made difficult by the lack of LTE (a consequence of the presence of
long-lived Mg 2p holes), and robust comparison to finite-temperature DFT
calculations is therefore impossible. Consequently, future experiments
will aim for a closer approach to local thermal equilibrium. This can be
achieved in two ways: first, by choosing a system free of long-lived
semicore excitations and second, by carrying out time-resolved
measurements using two-color pump-probe operations of the XFEL.

\emph{\\}

\section{References}

{[}1{]} M. Koenig \emph{et al.}, Plasma Physics and Controlled Fusion
\textbf{47}, B441 (2005).

{[}2{]} S. Atzeni and J. Meyer-ter-Vehn, \emph{The Physics of Inertial
Fusion: BeamPlasma Interaction, Hydrodynamics, Hot Dense Matter} (Oxford
University Press on Demand, 2004), Vol. 125.

{[}3{]} G. Faussurier, C. Blancard, P. Cossé, and P. Renaudin, Physics
of Plasmas \textbf{17}, 052707 (2010).

{[}4{]} A. Höll \emph{et al.}, High Energy Density Physics \textbf{3},
120 (2007).

{[}5{]} B. E. Warren, \emph{X-ray Diffraction} (Courier Corporation,
1969).

{[}6{]} J. Yano and V. K. Yachandra, Photosynthesis Research
\textbf{102}, 241 (2009).

{[}7{]} R. A. Valenza and G. T. Seidler, Physical Review B \textbf{93},
115135 (2016).

{[}8{]} T. Ma \emph{et al.}, Physical Review Letters \textbf{110},
065001 (2013).

{[}9{]} J. Chalupsky \emph{et al.}, Nuclear Instruments and Methods in
Physics Research Section A: Accelerators, Spectrometers, Detectors and
Associated Equipment \textbf{631}, 130 (2011).

{[}10{]} S. Herrmann \emph{et al.}, Nuclear Instruments and Methods in
Physics Research Section A: Accelerators, Spectrometers, Detectors and
Associated Equipment \textbf{718}, 550 (2013).

{[}11{]} P. Hart \emph{et al.}, in \emph{Proc. SPIE} \textbf{8504},
\emph{The CSPAD megapixel x-ray camera at LCLS,} 2012, p. 85040C.

{[}12{]} D. Damiani \emph{et al.}, Journal of Applied Crystallography
\textbf{49}, 672 (2016).

{[}13{]} O. R. Hoidn and G. T. Seidler, (in preparation).

{[}14{]} F. Salvat, J. M. Fernández-Varea, and J. Sempau, \textbf{7},
\emph{PENELOPE-2006: A code system for Monte Carlo simulation of
electron and photon transport,} OECD Publishing, Paris, 2006.

{[}15{]} J. Abdallah Jr, R. E. Clark, and R. D. Cowan, Theoretical
atomic physics code development I: CATS: Cowan Atomic Structure Code,
1988.

{[}16{]} O. Ciricosta \emph{et al.}, Nature Communications \textbf{7},
11713 (2016).

{[}17{]} H.-K. Chung, B. Cho, O. Ciricosta, S. Vinko, J. Wark, and R.
Lee, in \emph{AIP Conference Proceedings} \textbf{1811}, \emph{Atomic
processes modeling of X-ray free electron laser produced plasmas using
SCFLY code,} AIP Publishing, 2017, p. 020001.

{[}18{]} J. Hafner, Journal of computational chemistry \textbf{29}, 2044
(2008).

{[}19{]} A. L. Ankudinov, B. Ravel, J. J. Rehr, and S. D. Conradson,
Physical Review B \textbf{58}, 7565 (1998).

{[}20{]} F. Neese, JBIC Journal of Biological Inorganic Chemistry
\textbf{11}, 702 (2006).

{[}21{]} C. Caleman, N. Timneanu, A. V. Martin, H. O. Jonsson, A.
Aquila, A. Barty, H. A. Scott, T. A. White, and H. N. Chapman, Optics
Express \textbf{23}, 1213 (2015).

{[}22{]} C. Kittel, \emph{Introduction to Solid State Physics} (Wiley,
2005).

{[}23{]} F. Zamponi, P. Rothhardt, J. Stingl, M. Woerner, and T.
Elsaesser, Proceedings of the National Academy of Sciences \textbf{109},
5207 (2012).

{[}24{]} S. Stankic, M. Müller, O. Diwald, M. Sterrer, E. Knözinger, and
J. Bernardi, Angewandte Chemie International Edition \textbf{44}, 4917
(2005).

{[}25{]} S. Vinko, O. Ciricosta, and J. Wark, Nature communications
\textbf{5}, 3533 (2014).

{[}26{]} T. M. Nieuwenhuizen, Physical Review Letters \textbf{62}, 357
(1989).

{[}27{]} F. A. de Moura and M. L. Lyra, Physical Review Letters
\textbf{81}, 3735 (1998).

{[}28{]} O. Keski-Rahkonen and M. O. Krause, Atomic Data and Nuclear
Data Tables \textbf{14}, 139 (1974).

{[}29{]} J. C. Fuggle and J. E. Inglesfield, in \emph{Unoccupied
Electronic States} (Springer, 1992), pp. 1.




